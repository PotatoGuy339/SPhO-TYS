\documentclass{article}
\usepackage[utf8]{inputenc}
\usepackage{hyperref}
\usepackage{amsmath}
\usepackage{amsfonts}
\usepackage{graphicx}


\title{SPhO Ten Year Series (TYS) with Solutions: 2021 Questions}
\author{
    Solutions available on Victoris\\
    \texttt{victoris.org}
    % new collaborators add your name and contact here!
}

\date{\today}

\begin{document}
\maketitle
\section{2021}
\subsection{Question 1}

A parallel-plate capacitor is shown in the diagram with plate area of A = 10.5 cm$^2$ and plate separation $2d$ = 7.12 mm. The left half of the gap is filled with material of dielectric constant $\kappa_1$ = 21.0; the top of the right half is filled with material of dielectric constant $\kappa_2$ = 
42.0; the bottom of the right half is filled with material of dielectric constant $\kappa_3$ = 58.0. What 
is the capacitance? [4]
\subsection{Question 2}
In the rectangle shown, the sides have lengths 5.0 cm and 15 cm, q1 = -5.0 $\mu$C, and q2 = +2.0 $\mu$C. With  V = 0 at infinity, what is the electric potential at\\
(a) corner A and [1]\\
(b) corner B? [1]\\
(c) How much work is required to move a charge 
q3 = +3.0 $\mu$C from B to A along a diagonal of the rectangle? [1]\\
(d) Does this work increase or decrease the electric potential energy of the three-charge system? [1]\\
Is more, less, or the same work required if q3 is moved along a path that is\\
(e) inside the rectangle but not on a diagonal and [0.5]\\
(f) outside the rectangle? [0.5]    
\subsection{Question 3}

The conducting rod shown in the figure has length L and is being pulled along horizontal, frictionless conducting rails at a constant velocity $\Vec{v}$. The rails are connected at one end 
with a metal strip. A uniform magnetic field $\vec{B}$, directed out of the page, fills the region in 
which the rod moves. Assume that L = 10 cm, 
v = 5.0 m/s, and B = 1.2 T. What are the\\ 
(a) magnitude and [1]\\
(b) direction (up or down the page) of the emf induced in the rod? [0.5]\\
(c) magnitude and [0.5]\\     
(d) direction of the current in the conducting loop? [0.5]\\
Assume that the resistance of the rod is 0.40 $\Omega$ and that the resistance of the rails and metal strip is negligibly small. \\
(e) At what rate is thermal energy being generated in the rod? [1]\\    
(f) What external force on the rod is needed to maintain $\vec{v}$?[1.5]\\  
(g) At what rate does this force do work on the rod?[1] 

\end{document}
