\def\sphoyear{20XX}
\setcounter{section}{0}
\setcounter{solcounter}{0}

% Headers and Descriptions
\fancyhead[L]{\textbf{SPhO \sphoyear}} \fancyhead[R]{\textbf{Solutions}}


\begin{titlepage}
\centering

{\Huge\bfseries SPhO \sphoyear}

\vspace{1cm}

{\LARGE Solution Set}

\vspace{2cm}

{\Large Compiled by: Tan Chien Hao, \texttt{www.tchlabs.net}}

\vspace{2cm}

{\Large Edited/Proofread by: Keith Chan, Sun Yu Chieh}
%Collaborators please feel free to add on!

\vspace{2cm}

{\Large Solutions by: Tan Chien Hao, Keith Chan, Sun Yu Chieh}
%Collaborators please feel free to add on!

\vspace{2cm}

{\large Suggest changes at: \github}


\vfill

{\itshape Last edited: \today}
\end{titlepage}

(Copy Problem Set here)

% Question 1
\begin{solution}
    \begin{subsolution}
        The system of dielectrics can be considered as 3 separate capacitors.\\
        Total capacitance = $\frac{1}{\frac{1}{C_1}+\frac{1}{C_2+C_3}}$, where $C_1,C_2$ and $C_3$ are the capacitances of the three capacitors with dielectric constants$\kappa_1$,$\kappa_2$ and $\kappa_3$ respectively.\\
        By using the formula $C=\frac{\kappa\epsilon_0A}{d}$ where $A$ is the area of the parallel plates of the capacitor and $d$ is the perpendicular distance between the parallel plates, \\$$C_1=\frac{\kappa_1\epsilon_0(A/2)}{d}, C_2=\frac{\kappa_2\epsilon_0(A/2)}{d/2} and C_3=\frac{\kappa_3\epsilon_0(A/2)}{d/2}$$.
        bruh gg how to format this
        
        
        \end{subsolution}
    \begin{subsolution}
        Hello
    \end{subsolution}
\end{solution}